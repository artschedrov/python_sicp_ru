\section{Создание абстракций с помощью функций}
\subsection{Введение}

Компьютерная наука - это чрезвычайно обширная академическая дисциплина. Области глобально распределенных систем, искусственного интеллекта, робототехники, графики, безопасности, научных вычислений, компьютерной архитектуры и десятки развивающихся подобластей каждый год пополняются новыми методами и открытиями. Стремительный прогресс компьютерной науки не затронул практически ни одного аспекта человеческой жизни. Коммерция, коммуникация, наука, искусство, досуг и политика - все они были переосмыслены как компьютерные области.

Огромная производительность компьютерной науки возможна только потому, что она построена на элегантном и мощном наборе фундаментальных идей. Все вычисления начинаются с представления информации, определения логики ее обработки и разработки абстракций, которые управляют сложностью этой логики. Овладение этими основами потребует от нас понимания того, как именно компьютеры интерпретируют компьютерные программы и выполняют вычислительные процессы.

Эти фундаментальные идеи уже давно преподаются в Беркли по классическому учебнику Structure and Interpretation of Computer Programs (\url{https://mitpress.mit.edu/9780262510875/structure-and-interpretation-of-computer-programs/}) Гарольда Абельсона и Джеральда Джея Сассмана с Джули Сассман. Эти конспекты лекций в значительной степени заимствованы из этого учебника, который авторы любезно разрешили адаптировать и использовать повторно.

Начало нашего интеллектуального путешествия не требует пересмотра, и мы не должны ожидать, что он когда-либо понадобится.


\begin{quotation}
Мы собираемся изучить идею вычислительного процесса. Вычислительные процессы - это абстрактные существа, которые населяют компьютеры. По мере своего развития процессы манипулируют другими абстрактными вещами, называемыми данными. Развитие процесса направляется шаблоном правил, называемым программой. Люди создают программы для управления процессами. По сути, мы вызываем духов компьютера с помощью наших заклинаний.

Программы, которые мы используем для управления процессами, похожи на заклинания колдуна. Они тщательно составляются из символических выражений на заумных и эзотерических языках программирования, которые определяют задачи, которые мы хотим, чтобы выполняли наши процессы.

Вычислительный процесс в правильно работающем компьютере выполняет программы точно и аккуратно. Поэтому, подобно ученику колдуна, начинающие программисты должны научиться понимать и предвидеть последствия своих колдовских действий.

-Абельсон и Сассман, SICP (1993)
\end{quotation}

\subsubsection{Программирование на языке Python}
\begin{quotation}
Язык - это не столько то, что вы изучаете, сколько то, к чему вы приобщаетесь.

-Арика Окрент
\end{quotation}

Чтобы определить вычислительные процессы, нам нужен язык программирования, желательно такой, который смогут понять как многие люди, так и множество компьютеров. В этом курсе мы будем изучать язык Python.

Python - это широко распространенный язык программирования, который привлек энтузиастов из многих профессий: веб-программистов, игровых инженеров, ученых, академиков и даже разработчиков новых языков программирования. Изучая Python, вы присоединяетесь к сообществу разработчиков, насчитывающему миллион человек. Сообщества разработчиков - это чрезвычайно важные институты: их члены помогают друг другу решать проблемы, делятся своим кодом и опытом, а также коллективно разрабатывают программное обеспечение и инструменты. Благодаря своему вкладу преданные члены сообщества часто становятся знаменитостями и пользуются всеобщим уважением. Возможно, когда-нибудь и вы окажетесь в числе этих элитных Питонистов.

Сам язык Python - это продукт большого сообщества добровольцев (\url{https://www.python.org/community/}), которое гордится разнообразием своих участников. Язык был задуман и впервые реализован Гвидо ван Россумом (\url{https://en.wikipedia.org/wiki/Guido_van_Rossum}) в конце 1980-х годов. В первой главе его учебника по Python 3 (\url{https://docs.python.org/3/tutorial/appetite.html}) объясняется, почему Python так популярен среди множества существующих сегодня языков.

Python отлично подходит в качестве учебного языка, поскольку на протяжении всей своей истории разработчики Python подчеркивали возможность интерпретации кода Python человеком, что подкреплялось руководящими принципами Zen of Python (\url{https://peps.python.org/pep-0020/}) - красотой, простотой и читабельностью. Python особенно подходит для этого курса, потому что его широкий набор функций поддерживает множество различных стилей программирования, которые мы и будем изучать. Хотя не существует единого способа программирования на Python, есть набор соглашений, разделяемых сообществом разработчиков, которые облегчают процесс чтения, понимания и расширения существующих программ. Таким образом, сочетание гибкости и доступности Python позволяет студентам изучить множество парадигм программирования, а затем применить полученные знания в тысячах текущих проектов (\url{https://pypi.org}).

Эти материалы поддерживают дух SICP, знакомя с особенностями Python в сочетании с методами проектирования абстракций и строгой моделью вычислений. Кроме того, эти материалы содержат практическое введение в программирование на Python, включая некоторые расширенные возможности языка и наглядные примеры. Изучение Python будет происходить естественно по мере прохождения курса.

Однако Python - богатый язык с множеством функций и возможностей, и мы сознательно вводим их постепенно, по мере того как накладываем слой фундаментальных концепций компьютерной науки. Опытным студентам, которые хотят быстро вникнуть во все тонкости языка, мы рекомендуем прочитать книгу Марка Пилгрима "Погружение в Python 3" (\url{https://diveintopython3.net}), которая находится в свободном доступе в Интернете. Темы в этой книге существенно отличаются от тем данного курса, но книга содержит очень ценную практическую информацию по использованию языка Python. Предупреждаем: в отличие от этих материалов, "Погружение в Python 3" предполагает наличие значительного опыта программирования.

Лучший способ начать программировать на Python - это напрямую взаимодействовать с интерпретатором. В этом разделе описано, как установить Python 3, инициировать интерактивный сеанс с интерпретатором и начать программировать.

\subsubsection{Установка Python 3}
Как и любое другое замечательное программное обеспечение, Python имеет множество версий. В этом курсе будет использоваться самая последняя стабильная версия Python 3 (в настоящее время Python 3.2). На многих компьютерах уже установлены более старые версии Python, но они не подойдут для этого курса. Вы можете использовать любой компьютер для этого курса, но вам придется установить Python 3. Не волнуйтесь, Python бесплатен.

В книге Dive Into Python 3 есть подробные инструкции по установке для всех основных платформ (\url{https://diveintopython3.net/installing-python.html}). В этих инструкциях несколько раз упоминается Python 3.1, но вам лучше использовать Python 3.2 (хотя для данного курса различия незначительны). На всех учебных машинах кафедры EECS уже установлен Python 3.2.